%
% This is file `allow-frac-scale.tex'.
%
%   The following code tries to hack into the LaTeX2e's NFSS!!!
% In particular, the macro \empty@sfcnt, which is used by size
% functions <empty>, s, gen, sgen, genb, sgenb, subf and ssubf,
% is extended to support font scaling via fraction expressions.
% For example, s*[6/5] in addition to s*[1.2]. Of course, this
% requires eTeX's extensions of `expressions'.
%
% It should be input immediately before \documentclass:
%
%   %
% This is file `allow-frac-scale.tex'.
%
%   The following code tries to hack into the LaTeX2e's NFSS!!!
% In particular, the macro \empty@sfcnt, which is used by size
% functions <empty>, s, gen, sgen, genb, sgenb, subf and ssubf,
% is extended to support font scaling via fraction expressions.
% For example, s*[6/5] in addition to s*[1.2]. Of course, this
% requires eTeX's extensions of `expressions'.
%
% It should be input immediately before \documentclass:
%
%   %
% This is file `allow-frac-scale.tex'.
%
%   The following code tries to hack into the LaTeX2e's NFSS!!!
% In particular, the macro \empty@sfcnt, which is used by size
% functions <empty>, s, gen, sgen, genb, sgenb, subf and ssubf,
% is extended to support font scaling via fraction expressions.
% For example, s*[6/5] in addition to s*[1.2]. Of course, this
% requires eTeX's extensions of `expressions'.
%
% It should be input immediately before \documentclass:
%
%   %
% This is file `allow-frac-scale.tex'.
%
%   The following code tries to hack into the LaTeX2e's NFSS!!!
% In particular, the macro \empty@sfcnt, which is used by size
% functions <empty>, s, gen, sgen, genb, sgenb, subf and ssubf,
% is extended to support font scaling via fraction expressions.
% For example, s*[6/5] in addition to s*[1.2]. Of course, this
% requires eTeX's extensions of `expressions'.
%
% It should be input immediately before \documentclass:
%
%   \input{allow-frac-scale}
%   \documentclass[<options>]{<class>}
%
%%%%%%%%%%%%%%%%%%%%%%%%%%%%%%%%%%%%%%%%%%%%%%%%%%%%%%%%%%%%
%
\expandafter\edef\csname current@catcode\endcsname{\the\catcode`\@}
\catcode`\@=11 %
% Look ahead for a decimal or a fraction
\ifx\detect@decimal@or@frac\@undefined
  \def\detect@decimal@or@frac#1/#2\@nil{%
    \def\@tempa{#2}%
  }
\fi
% A general-purpose macro \AssignScaledResult{#1}{#2}{#3}{#4}
%   #1: One of \numexpr, \dimexpr, \glueexpr or \muexpr
%   #2: The target register to hold the scaled result
%   #3: Scale factor, either a decimal or a fraction
%   #4: The original pre-scaled register
\newcommand*\AssignScaledResult[4]{%
  \edef\@tempa{#3}%
  \expandafter\detect@decimal@or@frac\@tempa/\@nil
  \ifx\@tempa\@empty
    \expandafter\@firstoftwo
  \else
    \expandafter\@secondoftwo
  \fi
  {#2=#3#4}% scale by decimal
  {#2=#1#4*#3\relax}% scale by fraction
}
\RequirePackage{etoolbox}
% Redefine \set@fontsize
\patchcmd\set@fontsize
  {\baselineskip\f@linespread\baselineskip}
  {\AssignScaledResult\glueexpr\baselineskip\f@linespread\baselineskip}
  {\typeout{Support line spreading with fractions}}
  {\typeout{Couldn't patch \string\set@fontsize}}
% Redefine \empty@sfcnt
\patchcmd\empty@sfcnt
  {\@tempdimb \optional@arg\@tempdimb}
  {\AssignScaledResult\dimexpr\@tempdimb\optional@arg\@tempdimb}
  {\typeout{Support font scaling with fractions}}
  {\typeout{Couldn't patch \string\empty@sfcnt}}
\catcode`\@=\current@catcode\relax
\endinput
%   \documentclass[<options>]{<class>}
%
%%%%%%%%%%%%%%%%%%%%%%%%%%%%%%%%%%%%%%%%%%%%%%%%%%%%%%%%%%%%
%
\expandafter\edef\csname current@catcode\endcsname{\the\catcode`\@}
\catcode`\@=11 %
% Look ahead for a decimal or a fraction
\ifx\detect@decimal@or@frac\@undefined
  \def\detect@decimal@or@frac#1/#2\@nil{%
    \def\@tempa{#2}%
  }
\fi
% A general-purpose macro \AssignScaledResult{#1}{#2}{#3}{#4}
%   #1: One of \numexpr, \dimexpr, \glueexpr or \muexpr
%   #2: The target register to hold the scaled result
%   #3: Scale factor, either a decimal or a fraction
%   #4: The original pre-scaled register
\newcommand*\AssignScaledResult[4]{%
  \edef\@tempa{#3}%
  \expandafter\detect@decimal@or@frac\@tempa/\@nil
  \ifx\@tempa\@empty
    \expandafter\@firstoftwo
  \else
    \expandafter\@secondoftwo
  \fi
  {#2=#3#4}% scale by decimal
  {#2=#1#4*#3\relax}% scale by fraction
}
\RequirePackage{etoolbox}
% Redefine \set@fontsize
\patchcmd\set@fontsize
  {\baselineskip\f@linespread\baselineskip}
  {\AssignScaledResult\glueexpr\baselineskip\f@linespread\baselineskip}
  {\typeout{Support line spreading with fractions}}
  {\typeout{Couldn't patch \string\set@fontsize}}
% Redefine \empty@sfcnt
\patchcmd\empty@sfcnt
  {\@tempdimb \optional@arg\@tempdimb}
  {\AssignScaledResult\dimexpr\@tempdimb\optional@arg\@tempdimb}
  {\typeout{Support font scaling with fractions}}
  {\typeout{Couldn't patch \string\empty@sfcnt}}
\catcode`\@=\current@catcode\relax
\endinput
%   \documentclass[<options>]{<class>}
%
%%%%%%%%%%%%%%%%%%%%%%%%%%%%%%%%%%%%%%%%%%%%%%%%%%%%%%%%%%%%
%
\expandafter\edef\csname current@catcode\endcsname{\the\catcode`\@}
\catcode`\@=11 %
% Look ahead for a decimal or a fraction
\ifx\detect@decimal@or@frac\@undefined
  \def\detect@decimal@or@frac#1/#2\@nil{%
    \def\@tempa{#2}%
  }
\fi
% A general-purpose macro \AssignScaledResult{#1}{#2}{#3}{#4}
%   #1: One of \numexpr, \dimexpr, \glueexpr or \muexpr
%   #2: The target register to hold the scaled result
%   #3: Scale factor, either a decimal or a fraction
%   #4: The original pre-scaled register
\newcommand*\AssignScaledResult[4]{%
  \edef\@tempa{#3}%
  \expandafter\detect@decimal@or@frac\@tempa/\@nil
  \ifx\@tempa\@empty
    \expandafter\@firstoftwo
  \else
    \expandafter\@secondoftwo
  \fi
  {#2=#3#4}% scale by decimal
  {#2=#1#4*#3\relax}% scale by fraction
}
\RequirePackage{etoolbox}
% Redefine \set@fontsize
\patchcmd\set@fontsize
  {\baselineskip\f@linespread\baselineskip}
  {\AssignScaledResult\glueexpr\baselineskip\f@linespread\baselineskip}
  {\typeout{Support line spreading with fractions}}
  {\typeout{Couldn't patch \string\set@fontsize}}
% Redefine \empty@sfcnt
\patchcmd\empty@sfcnt
  {\@tempdimb \optional@arg\@tempdimb}
  {\AssignScaledResult\dimexpr\@tempdimb\optional@arg\@tempdimb}
  {\typeout{Support font scaling with fractions}}
  {\typeout{Couldn't patch \string\empty@sfcnt}}
\catcode`\@=\current@catcode\relax
\endinput
%   \documentclass[<options>]{<class>}
%
%%%%%%%%%%%%%%%%%%%%%%%%%%%%%%%%%%%%%%%%%%%%%%%%%%%%%%%%%%%%
%
\expandafter\edef\csname current@catcode\endcsname{\the\catcode`\@}
\catcode`\@=11 %
% Look ahead for a decimal or a fraction
\ifx\detect@decimal@or@frac\@undefined
  \def\detect@decimal@or@frac#1/#2\@nil{%
    \def\@tempa{#2}%
  }
\fi
% A general-purpose macro \AssignScaledResult{#1}{#2}{#3}{#4}
%   #1: One of \numexpr, \dimexpr, \glueexpr or \muexpr
%   #2: The target register to hold the scaled result
%   #3: Scale factor, either a decimal or a fraction
%   #4: The original pre-scaled register
\newcommand*\AssignScaledResult[4]{%
  \edef\@tempa{#3}%
  \expandafter\detect@decimal@or@frac\@tempa/\@nil
  \ifx\@tempa\@empty
    #2=#3#4%
  \else
    #2=#1#4*#3\relax
  \fi
}
\RequirePackage{etoolbox}
% Redefine \set@fontsize
\patchcmd\set@fontsize
  {\baselineskip\f@linespread\baselineskip}
  {\AssignScaledResult\glueexpr\baselineskip\f@linespread\baselineskip}
  {\typeout{Support line spreading with fractions}}
  {\typeout{Couldn't patch \string\set@fontsize}}
% Redefine \empty@sfcnt
\patchcmd\empty@sfcnt
  {\@tempdimb \optional@arg\@tempdimb}
  {\AssignScaledResult\dimexpr\@tempdimb\optional@arg\@tempdimb}
  {\typeout{Support font scaling with fractions}}
  {\typeout{Couldn't patch \string\empty@sfcnt}}
\catcode`\@=\current@catcode\relax
\endinput